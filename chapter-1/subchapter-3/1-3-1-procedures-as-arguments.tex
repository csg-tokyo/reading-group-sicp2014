\subsubsection{引数としてのプロシージャー}
以下の2つの例について考える.
%
\begin{lstlisting}[basicstyle=\footnotesize,title=$a$から$b$までの整数の和]
(define (sum-integers a b)
  (if (> a b)
      0
      (+ a (sum-integers (+ a 1) b))))
\end{lstlisting}
%
\begin{lstlisting}[basicstyle=\footnotesize,title=$\pi$の数列]
(define (sum-cubes a b)
  (if (> a b)
      0
      (+ (/ 1.0 (* a (+ a 2))) (pi-sum (+ a 4) b))))
\end{lstlisting}

1つ目の例は$a$から$b$までの整数の和を求める. 2つの目の例はパイに収束する.

以上の2つの例が非常に似ていおり, 変わるのは足される値と次の値を求める部分のみである.
関数を書く時にその部分を抽象化することによって, 関数の再利用性を上げることができる.
その抽象化は次のように, 以上の2つの操作を引数にすることによって行うことができる.
%
\begin{lstlisting}[basicstyle=\footnotesize,title=一般的な和の関数]
(define (sum term a next b)
  (if (> a b)
      0
      (+ (term a)
         (sum term (next a) next b))))
\end{lstlisting}

以上の関数を用いると, $a$から$b$までの整数の和を以下のように書きなおすことができる.

\begin{lstlisting}[basicstyle=\footnotesize,title=一般的な和を用いた$a$から$b$までの整数]
(define (inc n) (+ n 1))
(define (identity x) x)
(define (sum-integers a b) (sum identity a inc b))
\end{lstlisting}
