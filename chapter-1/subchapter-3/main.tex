\setcounter{section}{1}
\setcounter{subsection}{2}
\subsection{高階プロシージャーを用いた抽象化}
プロシージャーは抽象化の一種であり, 複雑な操作に名前を付けることで,
再利用することができる. ただし, プロシージャーの入力が常に数字であれば,
できることが限られる.

プロシージャーの入力としてプロシージャーを受け取ることで, 言語の表現力が
上がり, より効率よく抽象化できるようになる. そのようなプロシージャーは
高階プロシージャーと呼ぶ.

\subsubsection{引数としてのプロシージャ}
以下の2つの例について考える.
%
\begin{lstlisting}[basicstyle=\footnotesize,title=$a$から$b$までの整数の和]
(define (sum-integers a b)
  (if (> a b)
      0
      (+ a (sum-integers (+ a 1) b))))
\end{lstlisting}
%
\begin{lstlisting}[basicstyle=\footnotesize,title=$\pi$の数列]
(define (sum-cubes a b)
  (if (> a b)
      0
      (+ (/ 1.0 (* a (+ a 2))) (pi-sum (+ a 4) b))))
\end{lstlisting}

1つ目の例は$a$から$b$までの整数の和を求める. 2つの目の例はパイに収束する.

以上の2つの例が非常に似ていおり, 変わるのは足される値と次の値を求める部分のみである.
関数を書く時にその部分を抽象化することによって, 関数の再利用性を上げることができる.
その抽象化は次のように, 以上の2つの操作を引数にすることによって行うことができる.
\newpage
\begin{lstlisting}[basicstyle=\footnotesize,title=一般的な和の関数]
(define (sum term a next b)
  (if (> a b)
      0
      (+ (term a)
         (sum term (next a) next b))))
\end{lstlisting}

以上の関数を用いると, $a$から$b$までの整数の和を以下のように書きなおすことができる.
%
\begin{lstlisting}[basicstyle=\footnotesize,title=一般的な和を用いた$a$から$b$までの整数]
(define (inc n) (+ n 1))
(define (identity x) x)
(define (sum-integers a b) (sum identity a inc b))
\end{lstlisting}

%
\subsubsection{\lstinline{lambda}を用いたプロシージャの作成}
プロシージャをプロシージャに渡す時, 非常に単純なプロシージャを使うことが多い.
従って, 毎回定義するよりプロシージャをその場で作成できた方が使いやすい.
そうするために, 関数を返す\lstinline{lambda}というスペシャルフォームを用いることができる.
\lstinline{lambda}を用いて$\pi/8$に収束する数列を書き直すと,

\begin{lstlisting}[basicstyle=\footnotesize,title=一般的な和の関数と\lstinline{lambda}を用いた$\pi/8$に収束する数列]
(define (pi-sum a b)
  (sum (lambda (x) (/ 1.0 (* x (+ x 2))))
       a
       (lambda (x (+ x 4))
       b))
\end{lstlisting}

\paragraph{\lstinline{let}を用いたローカル変数の作成} \lstinline{lambda}は,
ローカル変数を作るために用いることができる. 例えば$f(x,y) = (1-y)(1+2x) - (1-y)4x^2$を次のように定義できる.

\begin{lstlisting}[basicstyle=\footnotesize]
(define (f x y)
  ((lambda (a b)
    (- (* b (+ 1 a)
       (* b (* a a)))))
   (* 2 x) (- 1 y)))
\end{lstlisting}

\lstinline{lambda}を用いる場合, 作ったプロシージャに変数を渡すことでローカル変数を作る.
ただし, ローカル変数を宣言することが多いので, より簡潔に定義できるためのスペシャルフォーム\lstinline{let}が存在する.
\lstinline{let}の場合は変数に値を割り当てて, \lstinline{let}のスコープでのローカル変数となる.
以上の例を\lstinline{let}を用いて書き直すと,

\begin{lstlisting}[basicstyle=\footnotesize]
(define (f x y)
  (let ((a (* 2 x))
        (b (- 1 y)))
    (- (* b (+ 1 a)
       (* b (* a a))))))
\end{lstlisting}

\lstinline{let}について, 以下の2つの点が重要である.

\begin{itemize}
\item 宣言される変数がローカルで, \lstinline{let}の外の環境に影響を与えない.
\item \lstinline{let}を使った変数の宣言の式は\lstinline{let}の外の環境で評価される.
\end{itemize}

%
\setcounter{subsubsection}{3}
\subsubsection{返り値としてのプロシージャー}
プロシージャーは引数としてだけでなく, 返り値として用いることによって
さらに表現できることを増やすことができる. 例えば, $f$関数があったとすると,
$x$と$f(x)$の平均を求める関数を以下のように定義できる.

\begin{lstlisting}[basicstyle=\footnotesize]
(define (average-damp f)
  (lambda (x) (average x (f x))))
\end{lstlisting}
\noindent
2乗を行う関数について
\lstinline[basicstyle=\footnotesize]{((average-damp square) 10) => 55}
のように用いることができる.

このように関数を返すような関数を用いることによって, 複雑な関数を簡単に作れるようになり,
その関数をまた他の関数の引数などで使うことができる.

%
\subsubsection*{抽象化と第一級プロシージャ}
プログラムを書く時, 抽象化できるところを見分け,
一般化のできるような抽象を作ることが必要である.
そのために, 高階プロシージャが非常に重要である.

プログラミング言語において, 要素の種類によって,
できることが限られることがある. そのような制限のない要素は「第一級」であるという.
具体的に, 以下の4つの点を満たすと第一級だという.

\begin{itemize}
\item 変数として定義できる.
\item 引数としてプロシージャに渡せる.
\item プロシージャの返り値として返せる.
\item データ構造に持たせることができる.
\end{itemize}

