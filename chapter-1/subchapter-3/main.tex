\setcounter{section}{1}
\setcounter{subsection}{2}
\subsection{高階プロシージャーを用いた抽象化}
プロシージャーは抽象化の一種であり, 複雑な操作に名前を付けることで, 
再利用することができる. ただし, プロシージャーの入力が常に数字であれば, 
できることが限られる. 

プロシージャーの入力としてプロシージャーを受け取ることで, 言語の表現力が
上がり, より効率よく抽象化できるようになる. そのようなプロシージャーは
高階プロシージャーと呼ぶ.

\subsubsection{引数としてのプロシージャー}
まず以下の2つの例について考える.

\begin{align*}
  f(a, b) &= \sum_{n = a}^{b} n    &\text{$a$から$b$までの整数の和}\\
  f(a, b) &= \sum_{n = a}^{b} n^3  &\text{$a$から$b$までの整数の立方数の和}
\end{align*}

この2つの例をそのままSchemeで書くと,

\begin{lstlisting}[basicstyle=\footnotesize,title=$a$から$b$までの整数の和]
(define (sum-integers a b)
  (if (> a b)
      0
      (+ a (sum-integers (+ a 1) b))))
\end{lstlisting}

\begin{lstlisting}[basicstyle=\footnotesize,title=$a$から$b$までの整数の立方数の和]
(define (sum-cubes a b)
  (if (> a b)
      0
      (+ (cube a) (sum-cubes (+ a 1) b))))
\end{lstlisting}
