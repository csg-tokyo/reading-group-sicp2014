\setcounter{section}{1}
\setcounter{subsection}{2}
\subsection{高階プロシージャを用いた抽象化}
プロシージャは抽象化の一種であり, 複雑な操作に名前を付けることで,
再利用することができる. ただし, プロシージャの入力が常に数字であれば,
できることが限られる.

プロシージャの入力としてプロシージャを受け取ることで, 言語の表現力が
上がり, より効率よく抽象化できるようになる. そのようなプロシージャは
高階プロシージャと呼ぶ.

\subsubsection{引数としてのプロシージャー}
まず以下の2つの例について考える.

\begin{align*}
  f(a, b) &= \sum_{n = a}^{b} n    &\text{$a$から$b$までの整数の和}\\
  f(a, b) &= \sum_{n = a}^{b} n^3  &\text{$a$から$b$までの整数の立方数の和}
\end{align*}

この2つの例をそのままSchemeで書くと,

\begin{lstlisting}[basicstyle=\footnotesize,title=$a$から$b$までの整数の和]
(define (sum-integers a b)
  (if (> a b)
      0
      (+ a (sum-integers (+ a 1) b))))
\end{lstlisting}

\begin{lstlisting}[basicstyle=\footnotesize,title=$a$から$b$までの整数の立方数の和]
(define (sum-cubes a b)
  (if (> a b)
      0
      (+ (cube a) (sum-cubes (+ a 1) b))))
\end{lstlisting}

%
\subsubsection{\lstinline{lambda}を用いたプロシージャーの作成}
プロシージャーをプロシージャーに渡す時, 非常に単純なプロシージャーを使うことが多い.
従って, 毎回定義するよりプロシージャーをその場で作成できた方が使いやすい.
そうするために, 関数を返す\lstinline{lambda}というスペシャルフォームを用いることができる.
\lstinline{lambda}を用いて$\pi$の数列を書き直すと,

\begin{lstlisting}[basicstyle=\footnotesize,title=一般的な和の関数と\lstinline{lambda}を用いた$\pi$の数列]
(define (pi-sum a b)
  (sum (lambda (x) (/ 1.0 (* x (+ x 2))))
       a
       (lambda (x (+ x 4))
       b))
\end{lstlisting}

%
\setcounter{subsubsection}{3}
\subsubsection{返り値としてのプロシージャ}
プロシージャは引数としてだけでなく, 返り値として用いることによって
さらに表現できることを増やすことができる. 例えば, $f$関数があったとすると,
$x$と$f(x)$の平均を求める関数を以下のように定義できる.

\begin{lstlisting}[basicstyle=\footnotesize]
(define (average-damp f)
  (lambda (x) (average x (f x))))
\end{lstlisting}
\noindent
2乗を行う関数について
\lstinline[basicstyle=\footnotesize]{((average-damp square) 10) => 55}
のように用いることができる.

このように関数を返すような関数を用いることによって, 複雑な関数を簡単に作れるようになり,
その関数をまた他の関数の引数などで使うことができる.

%
\subsubsection*{抽象化と第一級プロシージャ}
プログラムを書く時, 抽象化できるところを見分け,
一般化のできるような抽象を作ることが必要である.
そのために, 高階プロシージャが非常に重要である.

プログラミング言語において, 要素の種類によって,
できることが限られることがある. そのような制限のない要素は「第一級」であるという.
具体的に, 以下の4つの点を満たすと第一級だという.

\begin{itemize}
\item 変数として定義できる.
\item 引数としてプロシージャに渡せる.
\item プロシージャの返り値として返せる.
\item データ構造に持たせることができる.
\end{itemize}

