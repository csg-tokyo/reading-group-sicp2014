\setcounter{subsubsection}{3}
\subsubsection{返り値としてのプロシージャ}
プロシージャは引数としてだけでなく, 返り値として用いることによって
さらに表現できることを増やすことができる. 例えば, $f$関数があったとすると,
$x$と$f(x)$の平均を求める関数を以下のように定義できる.

\begin{lstlisting}[basicstyle=\footnotesize]
(define (average-damp f)
  (lambda (x) (average x (f x))))
\end{lstlisting}
\noindent
2乗を行う関数について
\lstinline[basicstyle=\footnotesize]{((average-damp square) 10) => 55}
のように用いることができる.

このように関数を返すような関数を用いることによって, 複雑な関数を簡単に作れるようになり,
その関数をまた他の関数の引数などで使うことができる.
