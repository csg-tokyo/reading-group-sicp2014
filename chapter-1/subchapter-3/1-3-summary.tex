\subsubsection*{抽象化と第一級プロシージャ}
プログラムを書く時, 抽象化できるところを見分け,
一般化のできるような抽象を作ることが必要である.
そのために, 高階プロシージャが非常に重要である.

プログラミング言語において, 要素の種類によって,
できることが限られることがある. そのような制限のない要素は「第一級」であるという.
具体的に, 以下の4つの点を満たすと第一級だという.

\begin{itemize}
\item 変数として定義できる.
\item 引数としてプロシージャに渡せる.
\item プロシージャの返り値として返せる.
\item データ構造に持たせることができる.
\end{itemize}
