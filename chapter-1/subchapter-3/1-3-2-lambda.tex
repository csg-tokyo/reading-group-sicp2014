\subsubsection{\lstinline{lambda}を用いたプロシージャの作成}
プロシージャをプロシージャに渡す時, 非常に単純なプロシージャを使うことが多い.
従って, 毎回定義するよりプロシージャをその場で作成できた方が使いやすい.
そうするために, 関数を返す\lstinline{lambda}というスペシャルフォームを用いることができる.
\lstinline{lambda}を用いて$\pi/8$に収束する数列を書き直すと,

\begin{lstlisting}[basicstyle=\footnotesize,title=一般的な和の関数と\lstinline{lambda}を用いた$\pi/8$に収束する数列]
(define (pi-sum a b)
  (sum (lambda (x) (/ 1.0 (* x (+ x 2))))
       a
       (lambda (x (+ x 4))
       b))
\end{lstlisting}

\paragraph{\lstinline{let}を用いたローカル変数の作成} \lstinline{lambda}は,
ローカル変数を作るために用いることができる. 例えば$f(x,y) = (1-y)(1+2x) - (1-y)4x^2$を次のように定義できる.

\begin{lstlisting}[basicstyle=\footnotesize]
(define (f x y)
  ((lambda (a b)
    (- (* b (+ 1 a)
       (* b (* a a)))))
   (* 2 x) (- 1 y)))
\end{lstlisting}

ローカル変数を宣言することが多いので, そのためのスペシャルフォーム\lstinline{let}が存在する.
以上の例を\lstinline{let}を用いて書き直すと,

\begin{lstlisting}[basicstyle=\footnotesize]
(define (f x y)
  (let ((a (* 2 x))
        (b (- 1 y)))
    (- (* b (+ 1 a)
       (* b (* a a))))))
\end{lstlisting}

\lstinline{let}について, 以下の2つの点が重要である.

\begin{itemize}
\item 宣言される変数がローカルで, \lstinline{let}の外の環境に影響を与えない.
\item \lstinline{let}を使った変数の宣言の式は\lstinline{let}の外の環境で評価される.
\end{itemize}
