\subsection{非決定性演算}
%
この節では, non deterministic(非決定性演算)という
プログラミングパラダイムを紹介する. これのパラダイムを
実装することによって, 自動探索ができるようになる.

非決定性演算は生成して値をテストするようなアプリケーションでは
便利である. 例えば, 2つの整数のリストから1つずつの整数を足した時に
素数になるような値を探す問題について考える. 今までの方法では,
すべてのペアを生成してそのフィルタを掛けるような実装が考えられる.

非決定性演算を用いると, 考え方が大きく違ってくる. ある結果に対して条件を
付けることができると考えると, 次のようなコードを考えられる.

\begin{lstlisting}[basicstyle=\footnotesize,caption=]
(define (prime-sum-pair list1 list2)
(let ((a (an-element-of list1))
      (b (an-element-of list2)))
  (require (prime? (+ a b)))
  (list a b)))
\end{lstlisting}

重要な概念としては, 2つ以上の結果が存在しうることである. 例えば,
\lstinline{an-element-of}は渡されたリストの任意の要素を返すことができる.
内部の評価機が自動的にあり得る値を選んでくれる.

これから実装する非決定性演算評価機はambと呼ぶ. その名前はambiguousという意味で
\lstinline{amb}という新しいスペシャルフォームを導入するからである.
amb評価機と以上のように定義された\lstinline{prime-sum-pair}で
次のような結果を得ることができる.

\begin{lstlisting}[basicstyle=\footnotesize,caption=]
(prime-sum-pair '(1 3 5 8) '(20 35 110))
(3 20)
\end{lstlisting}

また, ユーザが\lstinline{try-again}を打つことによって, 次の解が
表示される.
%
\subsubsection{Ambと探索}
非決定性演算をサポートするように, Schemeを\lstinline{amb}という新しい
スペシャルフォームで拡張する.

\lstinline{amb}は{\tt (amb $<e_1> <e_2> \cdots <e_n>$)}に対して
$n$個の値の中から任意の$<e_i>$を返す. 例えば,

\begin{lstlisting}[basicstyle=\footnotesize]
(list (amb 1 2 3) (amb 'a 'b))
\end{lstlisting}

は

\begin{lstlisting}[basicstyle=\footnotesize]
(1 a) (1 b) (2 a) (2 b) (3 a) (3 b)
\end{lstlisting}

のどれかの値を取りうる. 逆に, 引数の取らない\lstinline{amb}は
「失敗」だと考えられる. これを用いて, \lstinline{require}と\lstinline{an-element-of}を
次のように定義できる.

\begin{lstlisting}[basicstyle=\footnotesize]
(define (require p)
  (if (not p) (amb)))
(define (an-element-of items)
  (require (not (null? items)))
  (amb (car items) (an-element-of (cdr items))))
\end{lstlisting}

\lstinline{amb}の実装としてはいくつか考えられるが, 今回は実装としては,
あり得る1つ目の値を選び, その値でテストを行い, 失敗の場合は前の選択に戻り次の値
で試す. このような探索は深さ優先探索という.
%
\setcounter{subsubsection}{2}
\subsubsection{Amb評価機の実装}
普通のSchemeの式はの評価では, 値が返ってくる, 処理が終わらない, エラーが出るという
3つのケースがある. 非決定性演算を追加すると, 解が見つからずバックトラックするケース
が増えるので, 評価が複雑になる.
Amb評価機を作るために, プロシージャの実行を変える必要がある.

\paragraph{プロシージャと継続の実装}
以前作った評価機では, プロシージャを評価する時に環境を表す引数を受け取っていた.
Amb評価機の場合は, 環境引数以外にも2つの引数を渡す必要がある. その2つの引数は
継続プロシージャで, 成功の時に実行するプロシージャと失敗する時に実行するプロシージャである.

実行順番でいうと, あり得る任意の値が選択されて, その値と失敗のための時の新しい
失敗継続が成功継続に渡されていく. 失敗があった場合(あり得る値が選択できない場合)は,
失敗継続が呼び出されることによって, バックトラックを行う.

失敗継続については, 構築される場合はいくつかある.

\begin{itemize}
\item \lstinline{amb}式がある場合. \lstinline{amb}式がある場合, バックトラックできるように
  \lstinline{amb}式が構築される.
\item トップレベル. バックトラックができなくなった時, エラーを表示する.
\item 代入. バックトラックする時に代入を取り消すために.
\end{itemize}

また, 失敗継続が以下の2つのケースで呼び出される.

\begin{itemize}
\item \lstinline{(amb)}が実行される時.
\item ユーザは\lstinline{try-again}を実行した時.
\end{itemize}

さらに, 次の場合は失敗継続が連続で呼び出される.

\begin{itemize}
\item バックトラックする前に失敗継続でサイド・エフェクトの処理(例えば代入)を
  クリーンアップする場合は, クリーンアップが終わったあとに元々渡された失敗継続を
  呼び出しバックトラックの続きを行う.
\item 選択肢がなくなった時に, \lstinline{amb}に元々渡された失敗継続を呼び出すことによって
  失敗の前の選択点に戻る.
\end{itemize}
