\subsubsection{正規順と適用順}
章1.1では, Schemeが適用順序で評価することについて説明した.
従って, プロシージャを呼び出す時は, 引数がすべてすぐ評価される.
それに対して, 正規順序では, 引数の値必要がになった時に初めて
引数の式が評価される. そのような評価は遅延評価と呼ぶ.
例えば, 以下のコードについて考える.
%
\begin{lstlisting}[basicstyle=\footnotesize]
(define (try a b)
  (if (= a 0) 1 b))
\end{lstlisting}
%
以上のプロシージャで\lstinline{(try 0 (/ 1 0))}で呼び出したとすると,
適用順序の場合は, \lstinline{(/ 1 0)}が呼び出し時に評価され, エラーが生じてしまう.
それに対して, 正規順序と場合は, 引数\lstinline{b}の値が使われないため,
\lstinline{(/ 1 0)}が評価されることはないので, エラーが起きない.

プロシージャを呼び出す時に, プロシージャの実行が始まる前に評価される引数は
厳密であるという. プロシージャの実行が始まった時にまだ評価されていない引数は
非厳密であるという. 適用順序の場合は任意のプロシージャの引数はすべて
厳密であり, 正規順序の場合は任意のプロシージャの引数はすべて非厳密である.
