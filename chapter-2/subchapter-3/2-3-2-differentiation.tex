\subsubsection{記号微分}
%
記号を扱う例として, $ax^2 + bx + c$を$2ax + b$に
変えるような記号微分を行うプログラムを紹介する.

記号微分は以下の4つのルールで定義できる.

\begin{align*}
  \frac{\mathrm{d}c}{\mathrm{d}x} &= 0 \hspace{5mm} \text{$c$が定数もしくは$x$以外の変数} &
  %
  \frac{\mathrm{d}x}{\mathrm{d}x} &= 1\\
  %
  \frac{\mathrm{d}(u + v)}{\mathrm{d}x} &=
    \frac{\mathrm{d}u}{\mathrm{d}x} + \frac{\mathrm{d}v}{\mathrm{d}x} &
  %
  \frac{\mathrm{d}(uv)}{\mathrm{d}x} &=
    u\left(\frac{\mathrm{d}v}{\mathrm{d}x} \right) +
    v\left(\frac{\mathrm{d}u}{\mathrm{d}x} \right)
\end{align*}

以上の定義に従って実装するには, 単純式について,
変数であるかどうかと2つの変数について同じ変数であるかどう分かれば良い.
複合式について, 和か積の判別, 和と式の作成, 和と式からオペランドを
取り出せれば, 定義通りに実装可能である.

変数を記号として表すと, 単純式のための関数次のように定義できる.

\begin{lstlisting}[basicstyle=\footnotesize]
(define (variable? x) (symbol? x))
(define (same-variable? v1 v2)
  (and (variable? v1) (variable? v2) (eq? v1 v2)))
\end{lstlisting}

足し算と掛け算を次のように表すと,

\begin{lstlisting}[basicstyle=\footnotesize]
'(+ x 1)
'(* x 1)
\end{lstlisting}

\lstinline{ca(dd)r}で要素を取り出し, 記号を比べることで
積と和の判定などができる.

最終的に次のような実装を考えられる.

\begin{lstlisting}[basicstyle=\scriptsize]
(define (deriv exp var)
  (cond ((number? exp) 0)
        ((variable? exp)
         (if (same-variable? exp var) 1 0))
        ((sum? exp)
         (make-sum (deriv (addend exp) var)
                   (deriv (augend exp) var)))
        ((product? exp)
         (make-sum
          (make-product (multiplier exp)
                        (deriv (multiplicand exp) var))
          (make-product (deriv (multiplier exp) var)
                        (multiplicand exp))))
        (else
         (error "unknown expression type -- DERIV" exp))))
\end{lstlisting}
