\subsubsection{引用}
%
複合データに次のようなデータを表したいとする.
%
\begin{lstlisting}[basicstyle=\footnotesize,title=任意の記号の入ったデータ]
(a b c d)
((Norah 12) (Molly 9) (Anna 7))
\end{lstlisting}
%
しかしながら, \lstinline{(list a b)}と書くと,
\lstinline{a}と\lstinline{b}の
値が入っているリストになってしまう. 従って, 記号を扱うために,
データを引用できるような仕組みが必要である.

自然言語では, 引用を行う時に「"」で囲むことが多いが,
Schemeでは, 引用したい記号の前に\lstinline{`}と書くと,
次のオブジェクトが引用される. 空白や括弧を用いるため,
括弧閉じる必要性もない.

\begin{lstlisting}[basicstyle=\footnotesize,title=記号データの例]
(define a 1)
(define b 2)

(list a b)
=> (1 2)

(list 'a 'b)
=> (a b)

(car '(a b c))
=> a
\end{lstlisting}
