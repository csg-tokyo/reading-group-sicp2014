\subsubsection{タイプタグ}
以上の例では, どの実装を選んでも四則演算の関数
が使えることがわかったが, 実装を実際選ぶ必要が
なくて, 直交座標の表現も極座標の表現も共存させる
ことができる. ただし, 同じプログラムで共存させる
ために, どの表現になっているかが分かる必要がある.
そのために, 簡単な方法としてはデータにタグをつけて,
どの形式で入ってるかをデータと一緒に保存しておく.
Schemeでは, データのリストに記号を追加すること
によって簡単に実現できる.

コンストラクタを用いて複素数を作る時に次のように
タイプをつける.

\begin{lstlisting}[basicstyle=\scriptsize]
(define (make-from-mag-ang-rectangular r a)
  (attach-tag 'rectangular
              (cons (* r (cos a)) (* r (sin a)))))
\end{lstlisting}

セレクタでタグを使ってデータを取り出す.

\begin{lstlisting}[basicstyle=\scriptsize]
(define (real-part z)
  (cond ((rectangular? z)
         (real-part-rectangular (contents z)))
        ((polar? z)
         (real-part-polar (contents z)))
        (else (error "Unknown type -- REAL-PART" z))))
\end{lstlisting}

そうすることによって, 両方の表現方法が同じプログラムで
共存できる.
