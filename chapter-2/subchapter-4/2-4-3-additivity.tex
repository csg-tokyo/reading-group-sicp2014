\subsubsection{データ適従プログラミングと付加性}
以上用いたのはタイプ別処理で, 効率よくモジュール化する
ために良い方法であるが, 以上の実装では, 予めすべての
表現を知っておかなければいけないので, 付加性に欠けている.
また, 同じプログラムで2つの関数は同じ名前を持ってはいけないので,
名前が被らないように変える必要が出てしまう.

その対策として, データ適従プログラミングを用いることができる.
型と操作を以下の2次元配列として見ることができる.

\begin{table}[h]
  \centering
  \begin{tabular}{c|c|c}
    \multicolumn{1}{c}{} & \multicolumn{2}{c}{\hspace{-1cm}変数型}\\
    \multicolumn{1}{c}{} & 直交座標表現 & 極座標表現\\
    \cline{2-3}
    \ttfamily{real-part} & \ttfamily{real-part-polar} & \ttfamily{real-part-rectangular}\\
    \ttfamily{imag-part} & \ttfamily{imag-part-polar} & \ttfamily{imag-part-rectangular}\\
    \ttfamily{magnitude} & \ttfamily{magnitude-polar} & \ttfamily{magnitude-rectangular}\\
    \ttfamily{angle} & \ttfamily{angle-polar} & \ttfamily{angle-rectangular}
  \end{tabular}
\end{table}
