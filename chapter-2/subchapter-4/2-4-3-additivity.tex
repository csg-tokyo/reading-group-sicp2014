\subsubsection{データ適従プログラミングと付加性}
以上用いたのはタイプ別処理で, 効率よくモジュール化する
ために良い方法であるが, 以上の実装では, 予めすべての
表現を知っておかなければいけないので, 付加性に欠けている.
また, 同じプログラムで2つの関数は同じ名前を持ってはいけないので,
名前が被らないように変える必要が出てしまう.

その対策として, データ適従プログラミングを用いることができる.
型と操作を以下の2次元配列として見ることができる.

\begin{table}[!h]
  \centering
  \begin{tabular}{c|c|c}
    \multicolumn{1}{c}{} & \multicolumn{2}{c}{\hspace{-1cm}変数型}\\
    \multicolumn{1}{c}{} & 直交座標表現 & 極座標表現\\
    \cline{2-3}
    \ttfamily{real-part} & \ttfamily{real-part-polar} & \ttfamily{real-part-rectangular}\\
    \ttfamily{imag-part} & \ttfamily{imag-part-polar} & \ttfamily{imag-part-rectangular}\\
    \ttfamily{magnitude} & \ttfamily{magnitude-polar} & \ttfamily{magnitude-rectangular}\\
    \ttfamily{angle} & \ttfamily{angle-polar} & \ttfamily{angle-rectangular}
  \end{tabular}
\end{table}

この配列に関数を追加するための\lstinline{put}と関数を取り出すための
\lstinline{put}という関数があるとすると, 直交座標の内部実装を
次のように追加できる.

\begin{lstlisting}[basicstyle=\scriptsize]
(define (install-rectangular-package)
  ;; internal procedures
  (define (real-part z) (car z))
  (define (imag-part z) (cdr z))
  (define (make-from-real-imag x y) (cons x y))
  (define (magnitude z)
    (sqrt (+ (square (real-part z))
             (square (imag-part z)))))
  (define (angle z)
    (atan (imag-part z) (real-part z)))
  (define (make-from-mag-ang r a)
    (cons (* r (cos a)) (* r (sin a))))
  ;; interface to the rest of the system
  (define (tag x) (attach-tag 'rectangular x))
  (put 'real-part '(rectangular) real-part)
  (put 'imag-part '(rectangular) imag-part)
  (put 'magnitude '(rectangular) magnitude)
  (put 'angle '(rectangular) angle)
  (put 'make-from-real-imag 'rectangular
       (lambda (x y) (tag (make-from-real-imag x y))))
  (put 'make-from-mag-ang 'rectangular
       (lambda (r a) (tag (make-from-mag-ang r a))))
  'done)
\end{lstlisting}

同じように名前衝突なく極座標の表現も追加できる.
以上を用いるために, コンストラクタとセレクタを
次のように定義することができる.

\begin{lstlisting}[basicstyle=\scriptsize]
(define (apply-generic op . args)
  (let ((type-tags (map type-tag args)))
    (let ((proc (get op type-tags)))
      (if proc(apply proc (map contents args))
          (error
           "No method for these types -- APPLY-GENERIC"
           (list op type-tags))))))

(define (real-part z) (apply-generic 'real-part z))
(define (imag-part z) (apply-generic 'imag-part z))
(define (magnitude z) (apply-generic 'magnitude z))
(define (angle z) (apply-generic 'angle z))

(define (make-from-real-imag x y)
  ((get 'make-from-real-imag 'rectangular) x y))
(define (make-from-mag-ang r a)
((get 'make-from-mag-ang 'polar) r a))
\end{lstlisting}

ここでは, 一般的なセレクタ作って, \lstinline{apply-generic}を
使うことによって既存の関数を動的に用いる. この方法では,
動的に新しい表現を追加することもできるし, 名前の衝突も
起こることがない. 以上のアプローチを用いるとプログラムに
より単純に新しいモジュールを追加することができる.

\paragraph{メッセージパッシング}
以上では, それぞれの操作は表の1行になっており,
型からふさわしい関数を取り出すような形式になっている.

その代わりに, 次のようにオブジェクトデータに機能を持たせ,
受け取った関数名によって挙動を決める. その方法は
メッセージパッシングと呼ぶ.

\begin{lstlisting}[basicstyle=\footnotesize]
(define (make-from-real-imag x y)
  (define (dispatch op)
    (cond ((eq? op 'real-part) x)
          ((eq? op 'imag-part) y)
          ((eq? op 'magnitude)
           (sqrt (+ (square x) (square y))))
          ((eq? op 'angle) (atan y x))
          (else
           (error "Unknown op -- MAKE-FROM-REAL-IMAG" op))))
  dispatch)
\end{lstlisting}

以上では, \lstinline{make-from-real-imag}は関数を返し,
その関数は渡された「メッセージ」によって結果を返す.

章3から, メッセージパッシングはシミュレーションプログラムに
おいてどう役に立つのかについてより詳しく説明する.
