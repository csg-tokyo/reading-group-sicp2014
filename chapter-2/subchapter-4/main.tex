\subsection{抽象データの複数の表現方法}
%
前の章では, 抽象化の壁について話して,
抽象データの表現について説明した. 有理数の場合,
\lstinline{make-rat}, \lstinline{numer}, \lstinline{denom}
で抽象データを表現し, 実装を意識せずに扱えるようにした.

しかしながら, 概念によっては表現方法が複数存在することも多く,
使い方によって選ぶ表現方法が変わることもある. また, 同じプログラムは
長く用いられることも多いので, 最初から1つの表現方法を決めることが
困難である. 従って, プログラムの既存のモジュールを書き換えずに
新しいモジュールを追加できるような仕組みが重要である.

そのために, この章では複数の表現方法で表現されているデータ
の扱えるジェネリックプロシージャ, 表現方法を表すための
タイプタグを紹介する. また, モジュールの追加が簡単にできる
データダイレクテッドプログラミングについて説明する.
%
\subsubsection{複素数の表現}

