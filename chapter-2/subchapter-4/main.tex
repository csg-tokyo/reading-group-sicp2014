\subsection{抽象データの多重表現}
%
前の章では, 抽象化の壁について話して,
抽象データの表現について説明した. 有理数の場合,
\lstinline{make-rat}, \lstinline{numer}, \lstinline{denom}
で抽象データを表現し, 実装を意識せずに扱えるようにした.

しかしながら, 概念によっては表現方法が複数存在することも多く,
使い方によって選ぶ表現方法が変わることもある. また, 同じプログラムは
長く用いられることも多いので, 最初から1つの表現方法を決めることが
困難である. 従って, プログラムの既存のモジュールを書き換えずに
新しいモジュールを追加できるような仕組みが重要である.

そのために, この章では複数の表現方法で表現されているデータ
の扱えるジェネリックプロシージャ, 表現方法を表すための
タイプタグを紹介する. また, モジュールの追加が簡単にできる
データ適従プログラミングについて説明する.
%
\subsubsection{複素数の表現}
複素数は主な2つの方法で表現する.

\begin{itemize}
  \item 直交座標
    \[ z = x + iy \]
  \item 極座標
    \[ z = re^{i\phi}\]
\end{itemize}

この2つの表現から, 複素数の足し算と掛け算を次のように書くことができる.
\begin{align*}
  z_1 + z_2 &= (x_1 + x_2) + (y_1 + y_2)i\\
  z_1 \cdot z_2 &= (r_1 \cdot r_2) e^{i(\phi_1 + \phi_2)}
\end{align*}

\lstinline{real-part} ($x$), \lstinline{image-part} ($y$),
\lstinline{magnitude} ($r$), \lstinline{angle} ($\phi$)
というセレクタと, \\ \lstinline{make-from-real-image}と
\lstinline{make-from-mag-ang}というコンストラクタが
あったとすると, ある複素数$z$に対して

\begin{lstlisting}[basicstyle=\footnotesize]
(make-from-real-imag (real-part z) (imag-part z))
(make-from-mag-ang (magnitude z) (angle z))
\end{lstlisting}

は等しい複素数を返す.

そのコンストラクタとセレクタを用いて複素数の四則演算
を次のように定義できる.

\begin{lstlisting}[basicstyle=\footnotesize]
(define (add-complex z1 z2)
  (make-from-real-imag (+ (real-part z1) (real-part z2))
                       (+ (imag-part z1) (imag-part z2))))
(define (sub-complex z1 z2)
  (make-from-real-imag (- (real-part z1) (real-part z2))
                       (- (imag-part z1) (imag-part z2))))
(define (mul-complex z1 z2)
  (make-from-mag-ang (* (magnitude z1) (magnitude z2))
                     (+ (angle z1) (angle z2))))
(define (div-complex z1 z2)
  (make-from-mag-ang (/ (magnitude z1) (magnitude z2))
                     (- (angle z1) (angle z2))))
\end{lstlisting}
%
内部の複素数の表現方法として直交座標を選んでも極座標を選んでも,
セレクタとコンストラクタが正しく定義されていれば,
どちらでも以上の関数を使うことができる.

%
\subsubsection{タイプタグ}
以上の例では, どの実装を選んでも四則演算の関数
が使えることがわかったが, 実装を実際選ぶ必要が
なくて, 直交座標の表現も極座標の表現も共存させる
ことができる. ただし, 同じプログラムで共存させる
ために, どの表現になっているかが分かる必要がある.
そのために, 簡単な方法としてはデータにタグをつけて,
どの形式で入ってるかをデータと一緒に保存しておく.
Schemeでは, データのリストに記号を追加すること
によって簡単に実現できる.

コンストラクタを用いて複素数を作る時に次のように
タイプをつける.

\begin{lstlisting}[basicstyle=\scriptsize]
(define (make-from-mag-ang-rectangular r a)
  (attach-tag 'rectangular
              (cons (* r (cos a)) (* r (sin a)))))
\end{lstlisting}

セレクタでタグを使ってデータを取り出す.

\begin{lstlisting}[basicstyle=\scriptsize]
(define (real-part z)
  (cond ((rectangular? z)
         (real-part-rectangular (contents z)))
        ((polar? z)
         (real-part-polar (contents z)))
        (else (error "Unknown type -- REAL-PART" z))))
\end{lstlisting}

そうすることによって, 両方の表現方法が同じプログラムで
共存できる.

%
\subsubsection{データ適従プログラミングと付加性}
以上用いたのはタイプ別処理で, 効率よくモジュール化する
ために良い方法であるが, 以上の実装では, 予めすべての
表現を知っておかなければいけないので, 付加性に欠けている.
また, 同じプログラムで2つの関数は同じ名前を持ってはいけないので,
名前が被らないように変える必要が出てしまう.

その対策として, データ適従プログラミングを用いることができる.
型と操作を以下の2次元配列として見ることができる.

\begin{table}[h]
  \centering
  \begin{tabular}{c|c|c}
    \multicolumn{1}{c}{} & \multicolumn{2}{c}{\hspace{-1cm}変数型}\\
    \multicolumn{1}{c}{} & 直交座標表現 & 極座標表現\\
    \cline{2-3}
    \ttfamily{real-part} & \ttfamily{real-part-polar} & \ttfamily{real-part-rectangular}\\
    \ttfamily{imag-part} & \ttfamily{imag-part-polar} & \ttfamily{imag-part-rectangular}\\
    \ttfamily{magnitude} & \ttfamily{magnitude-polar} & \ttfamily{magnitude-rectangular}\\
    \ttfamily{angle} & \ttfamily{angle-polar} & \ttfamily{angle-rectangular}
  \end{tabular}
\end{table}

