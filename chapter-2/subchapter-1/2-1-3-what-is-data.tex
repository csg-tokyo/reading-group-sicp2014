\subsubsection{データとは}
以上でセレクターとコンストラクタで具体的なデータを抽象的なデータに
変えることがわかったが, 抽象的なデータについて定義を固める必要がある.

\noindent
分数の例において, セレクターについて考える時,
${}^{\forall} ($\lstinline{n}$, $\lstinline{d}$)
\in (\mathbb{N}\times \mathbb{N}\backslash \{0\})$
に対して, 次の関係が満たされる必要がある.
\[
  \text{\lstinline{x = (make-rat n d)}} \Leftrightarrow
  \frac{\text{\lstinline{numer x}}}{\text{\lstinline{denom x}}}
  = \frac{\text{\lstinline{n}}}{\text{\lstinline{d}}}
\]

一般的にデータは正しく表現されるための条件満たしているセレクターとコンストラクタの集合であると言える.
