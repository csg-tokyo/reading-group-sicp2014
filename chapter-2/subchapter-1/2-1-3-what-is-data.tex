\subsubsection{データとは}
以上で,セレクタとコンストラクタを用いて具体的なデータを抽象的なデータに
変換できることがわかったが, 抽象的なデータについて定義を固める必要がある.

\noindent
先ほどまでの例において、有理数はコンストラクタ(\lstinline{make-rat})と
セレクタ(\lstinline{numer}、\lstinline{denom})を用いて扱うことができた。
しかし、任意の三つの手続きの組が、有理数の実装に使えるわけではなく、
${}^{\forall} ($\lstinline{n}$, $\lstinline{d}$)
\in (\mathbb{N}\times \mathbb{N}\backslash \{0\})$
に対して, 次の関係が満たされる必要がある.
\[
  \text{\lstinline{x = (make-rat n d)}} \Leftrightarrow
  \frac{\text{\lstinline{numer x}}}{\text{\lstinline{denom x}}}
  = \frac{\text{\lstinline{n}}}{\text{\lstinline{d}}}
\]

従って、データとはコンストラクタとセレクタの集合と、それらのプロシージャ
がデータを正しく表現するための条件によって定義される。

この定義は、有理数のようなハイレベルなデータだけではなく、ローレベルな
データについても有効である。

例えば、有理数の実装に用いた「ペア」というデータ構造について考えてみる。
我々は今まで言語実装のコンストラクタ(\lstinline{cons})と
セレクタ(\lstinline{car}、\lstinline{cdr})を用いて扱ってきた。
重要なのはそれらのプロシージャがどう実装されているかではなく、
以下のような条件を満たしているかどうかである。

\[
  \text{\lstinline{z = (cons x y)}} \Rightarrow
  \text{\lstinline{(car z) = x}} \ \text{\lstinline{and}} \ \text{\lstinline{(cdr z) = y}}
\]

つまり、これらの条件を満たせば、言語レベルのデータ構造を用いずとも、
プロシージャのみを用いて実装することが可能である。

以下に、プロシージャを用いた有理数のためのコンストラクタ及びセレクタの定義を示す。

\begin{lstlisting}[basicstyle=\footnotesize,title=プロシージャによるペアの実装]
(define (cons x y)
  (define (dispatch m)
    (cond ((= m 0) x)
          ((= m 1) y)
          (else (error ``Argument not 0 or 1 -- CONS'' m))))

(define (car z) (z 0))

(define (cdr z) (z 1))
\end{lstlisting}
\vspace{5mm}

これにより、プロシージャを用いて、先に述べた条件を満たすコンストラクタ
とセレクタを定義することができることをわかった(ただし、ほとんどの言語では
効率の観点からペアを直接実装している)。この例は、プロシージャを第一級として
扱える能力が、自動的に、合成データをも表現できる能力となることも示している。
このプログラミング手法はしばしばメッセージパッシングスタイルと呼ばれ、
モデリングとシミュレーションについて論ずる第3章において基礎的なツール
として使われることになる。
