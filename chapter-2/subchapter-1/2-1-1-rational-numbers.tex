\subsubsection{有理数での算術}
有理数を使って四則演算を実装したいとする. 分数を表す方法がある前提で考えて,
分子と分母から分数を返す\lstinline{make-rat}, 分数から分子を返す\lstinline{numer},
分数から分母を返す\lstinline{denom}という3つのプロシージャが存在するとする.

分数の四則演算は

\begin{align*}
  &\frac{n_1}{d_1} + \frac{n_2}{d_2}  = \frac{n_1d_2 + n_2d_1}{d_1d_2}
  & \frac{n_1}{d_1}\cdot \frac{n_2}{d_2} = \frac{n_1n_2}{d_1d_2}\\
  &\frac{n_1}{d_1} - \frac{n_2}{d_2} = \frac{n_1d_2 - n_2d_1}{d_1d_2}
  & \frac{n_1/d_2}{n_2/d_2} = \frac{n_1d_2}{d_1n_2}\\
  &\frac{n_1}{d_1} = \frac{n_2}{d_2} \Leftrightarrow n_1d_2 = n_2d_1
\end{align*}
\noindent
のように定義できるので, 以上の3つのプロシージャがあれば, 問題なくそれぞれの
演算を問題なく実装できる. 例えば, 足し算は以下のように実装できる.

\begin{lstlisting}[basicstyle=\footnotesize,title=分数の足し算]
(define (add-rat x y)
  (make-rat (+ (* (numer x) (denom y))
               (* (numer y) (denom x)))
            (* (denom x) (denom y))))
\end{lstlisting}
\vspace{5mm}

Lispにはペアというデータ構造が存在しており, 2つのデータを1つの構造として表現できる.
\lstinline{cons}で2つの要素から1つのペアを作ることができ, \lstinline{car}で
1つ目の要素を取り出し, \lstinline{cdr}で2つ目の要素を取り出す.
ペアを用いると, 分数を自然に表現することができる.

\begin{lstlisting}[basicstyle=\footnotesize,title=分数の表現]
(define (make-rat n d) (cons n d))
(define (numer x) (car x))
(define (denom x) (cdr x))
\end{lstlisting}
