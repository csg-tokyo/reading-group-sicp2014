\setcounter{section}{2}
\setcounter{subsection}{0}
%
\subsection{データの抽象化入門}
プロシージャを作ることによって, 手続きを合成するだけでなく,
抽象化として見ることができる.

データの合成についても同じことができ, それをデータの抽象化という.
プログラムにおいて, 抽象的にデータを扱うというのは,
可能な限りデータがどう表現されているかに関係なく扱うことである.
それに対して, データが具体的にどう実装されているかはプログラムと関係なく定義されているものである.

具体的なデータと抽象的なデータの間に変換するために, セレクタとコンストラクタを用いる.
%
\subsubsection{有理数での算術}
有理数を使って四則演算を実装したいとする. 分数を表す方法がある前提で考えて,
分子と分母から分数を返す\lstinline{make-rat}, 分数から分子を返す\lstinline{numer},
分数から分母を返す\lstinline{denom}という3つのプロシージャが存在するとする.

分数の四則演算は

\begin{align*}
  &\frac{n_1}{d_1} + \frac{n_2}{d_2}  = \frac{n_1d_2 + n_2d_1}{d_1d_2}
  & \frac{n_1}{d_1}\cdot \frac{n_2}{d_2} = \frac{n_1n_2}{d_1d_2}\\
  &\frac{n_1}{d_1} - \frac{n_2}{d_2} = \frac{n_1d_2 - n_2d_1}{d_1d_2}
  & \frac{n_1/d_2}{n_2/d_2} = \frac{n_1d_2}{d_1n_2}\\
  &\frac{n_1}{d_1} = \frac{n_2}{d_2} \Leftrightarrow n_1d_2 = n_2d_1
\end{align*}
\noindent
のように定義できるので, 以上の3つのプロシージャがあれば, 問題なくそれぞれの
演算を問題なく実装できる. 例えば, 足し算は以下のように実装できる.

\begin{lstlisting}[basicstyle=\footnotesize,title=分数の足し算]
(define (add-rat x y)
  (make-rat (+ (* (numer x) (denom y))
               (* (numer y) (denom x)))
            (* (denom x) (denom y))))
\end{lstlisting}
\vspace{5mm}

Lispにはペアというデータ構造が存在しており, 2つのデータを1つの構造として表現できる.
\lstinline{cons}で2つの要素から1つのペアを作ることができ, \lstinline{car}で
1つ目の要素を取り出し, \lstinline{cdr}で2つ目の要素を取り出す.
ペアを用いると, 分数を自然に表現することができる.

\begin{lstlisting}[basicstyle=\footnotesize,title=分数の表現]
(define (make-rat n d) (cons n d))
(define (numer x) (car x))
(define (denom x) (cdr x))
\end{lstlisting}

%
\subsubsection{抽象化の壁}
データを扱う時、複数の抽象化のレイヤに分けることが出来る.
分数の例で考えると, 以下の図のようなレイヤが考えられる.
\begin{figure}[ht]
  \centering
  \includegraphics[width=8cm,height=6cm]{imgs/abstraction-barrier.png}
  \caption{\label{fig:abstraction-barrier}有理数における抽象化の壁}
\end{figure}

ここでは抽象化のレベルが4つで分かれている. 最も上のレイヤは分数を扱うプログラムが使うもので,
実際分数はどう表現されているかも, どう四則演算が定義されているのかが分からなくても四則演算が
できるレベルである. その下のレイヤでは, 実際の四則演算の実装である. そこでは分数のセレクタと
コンストラクタを用いるが, その実装に依存しない. その下のレイヤはセレクタとコンストラクタの実装で,
ペアを用いるが,今回もその実装に依存しない. 最も下のレイヤはペアの実装である.

そのレイヤ分けの主な利点はプログラムの保守性と柔軟性の向上である.
下のレイヤの実装に依存しないので, データをどう表現するかが変わらない限り,
実装が変わっても影響が受けない.
図\ref{fig:abstraction-barrier}で有理数における抽象化の壁の例を示す.

%
\subsubsection{データとは}
以上でセレクタとコンストラクタで具体的なデータを抽象的なデータに
変えることがわかったが, 抽象的なデータについて定義を固める必要がある.

\noindent
分数の例において, セレクタについて考える時,
${}^{\forall} ($\lstinline{n}$, $\lstinline{d}$)
\in (\mathbb{N}\times \mathbb{N}\backslash \{0\})$
に対して, 次の関係が満たされる必要がある.
\[
  \text{\lstinline{x = (make-rat n d)}} \Leftrightarrow
  \frac{\text{\lstinline{numer x}}}{\text{\lstinline{denom x}}}
  = \frac{\text{\lstinline{n}}}{\text{\lstinline{d}}}
\]

一般的にデータは正しく表現されるための条件満たしているセレクタとコンストラクタの集合であると言える.

