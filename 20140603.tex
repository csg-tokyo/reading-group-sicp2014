\documentclass[a4paper,10pt,fleqn]{article}
\ifdefined\print
  \usepackage[print]{sicpsetup}
\else
  \usepackage{sicpsetup}
\fi
\date{\today}

%
\author{Daniel Perez}
\sicptitle{4.2と4.3}
%
\begin{document}
\maketitle
%
\section*{概要}
%
前の章では, Lispプログラムの評価機を作った. 今回は, その評価機のいくつかの点を
変更して, 普通のLispの仕様を2つの方法で拡張・変更する.
その2つの方法について説明する.

1つ目の方法は, 評価順番を変える方法である. 普通のSchemeは適用順であるが,
今回遅延評価という概念を取り入れることによって, 正規順序評価に変える方法を紹介する.

もう1つの方法では, 普通のSchemeからもう少し離れて, 評価機に非決定性演算を追加する.
非決定性演算は評価機で自動的に探索を行うような機能で, 後半の部分ではその機能と
実装方法について説明する.


\setcounter{section}{4}
\setcounter{subsection}{1}
\subsection{遅延評価}
%
\subsubsection{正規順と適用順}
章1.1では, Schemeが適用順序で評価することについて説明した.
従って, プロシージャを呼び出す時は, 引数がすべてすぐ評価される.
それに対して, 正規順序では, 引数の値必要がになった時に初めて
引数の式が評価される. そのような評価は遅延評価と呼ぶ.
例えば, 以下のコードについて考える.
%
\begin{lstlisting}[basicstyle=\footnotesize]
(define (try a b)
  (if (= a 0) 1 b))
\end{lstlisting}
%
以上のプロシージャで\lstinline{(try 0 (/ 1 0))}で呼び出したとすると,
適用順序の場合は, \lstinline{(/ 1 0)}が呼び出し時に評価され, エラーが生じてしまう.
それに対して, 正規順序と場合は, 引数\lstinline{b}の値が使われないため,
\lstinline{(/ 1 0)}が評価されることはないので, エラーが起きない.

プロシージャを呼び出す時に, プロシージャの実行が始まる前に評価される引数は
厳密であるという. プロシージャの実行が始まった時にまだ評価されていない引数は
非厳密であるという. 適用順序の場合は任意のプロシージャの引数はすべて
厳密であり, 正規順序の場合は任意のプロシージャの引数はすべて非厳密である.

%
\subsubsection{遅延評価を用いたインタープリタ}
遅延評価に変えるには, プロシージャを適用する時にどの
引数を評価する必要があり, どの引数はまだ評価しないでいいのかを
判定する必要がある. そして, まだ評価しない引数については,
あとで評価するための情報を持ったthunkに変える. また,
そのthunkの値が必要になった時は, thunkを評価することは
forcingという. 最後に, インタープリターを作る時に
一度評価したthunkの値を保存して, 次にそのthunkを評価する時は,
保存した値を使うか, もう一度評価するかを決める必要がある.
今回は最初に評価した結果を保存し使いまわす設計を選ぶ.

\paragraph{インタープリタの変更}
遅延評価の場合は, 元々の評価方法と主な違いは適用の扱いにある.

元々の評価方法では, 次のコードを用いて適用を処理していた.

\begin{lstlisting}[basicstyle=\footnotesize]
((application? exp)
   (apply (eval (operator exp) env)
          (list-of-values (operands exp) env)))
\end{lstlisting}

以上の実装では, プロシージャを適用する前に引数が評価されるので,
変える必要がある.

\begin{lstlisting}[basicstyle=\footnotesize,caption=]
((application? exp)
 (apply (actual-value (operator exp) env)
        (operands exp)
        env))
\end{lstlisting}
変更としては, オペレータに対して\lstinline{eval}の代わりに
thunkを考慮してくれる\lstinline{actual-value}を呼び出し,
オペランドに対して, 評価するための\lstinline{list-of-values}を適用しなくなる.

他の重要な変更追加を以下に述べる.

\begin{lstlisting}[basicstyle=\footnotesize]
(define (actual-value exp env)
  (force-it (eval exp env)))

(define (force-it obj)
  (cond ((thunk? obj)
         (let ((result (actual-value
                        (thunk-exp obj)
                        (thunk-env obj))))
           (set-car! obj 'evaluated-thunk)
           (set-car! (cdr obj) result)
           (set-cdr! (cdr obj) '())
           result))
        ((evaluated-thunk? obj)
         (thunk-value obj))
        (else obj)))

(define (delay-it exp env)
  (list 'thunk exp env))

(define (apply procedure arguments env)
  (cond ((primitive-procedure? procedure)
         (apply-primitive-procedure
          procedure
          (list-of-arg-values arguments env)))
        ((compound-procedure? procedure)
         (eval-sequence
          (procedure-body procedure)
          (extend-environment
           (procedure-parameters procedure)
           (list-of-delayed-args arguments env)
           (procedure-environment procedure))))
        (else
         (error
          "Unknown procedure type -- APPLY" procedure))))
\end{lstlisting}

以上の実装では, thunkをリストとして表し, 評価されていないものに関して,
\lstinline{thunk}というラベルをつけて, 評価されたものに関して,
\lstinline{evaluated-thunk}というラベルをつける.
値を評価する時に, thunkでない場合はもう評価された値だと見てそのまま返し,
thunkの場合は評価済みかどうかを確認し, すでに評価されていれば値を再利用し,
まだ評価されていなければ, 評価してラベルを変え, 値を保存する.
最後に, \lstinline{apply}プロシージャでは, 入る\lstinline{arguments}は
評価されてない前提になるので, そこでケース2つで分ける.
\lstinline{primitive-procedure}の場合は, オペランドの値がすぐ必要になるので,
\lstinline{list-of-arg-values}でそれぞれのオペランドをを評価する.
\lstinline{compound-procedure}の場合は, 値がすぐに必要にならないので,
評価をせずに, \lstinline{list-of-delayed-args}を用いてそれぞれの
オペランドに\lstinline{delay-it}を適用してthunkのリストに変える.

%
% \setcounter{section}{2}
\setcounter{subsection}{0}
%
\subsection{データの抽象化入門}
プロシージャを作ることによって, 手続きを合成するだけでなく,
抽象化として見ることができる.

データの合成についても同じことができ, それをデータの抽象化という.
プログラムにおいて, 抽象的にデータを扱うというのは,
可能な限りデータがどう表現されているかに関係なく扱うことである.
それに対して, データが具体的にどう実装されているかはプログラムと関係なく定義されているものである.

具体的なデータと抽象的なデータの間に変換するために, セレクタとコンストラクタを用いる.
%
\subsubsection{有理数での算術}
有理数を使って四則演算を実装したいとする. 分数を表す方法がある前提で考えて,
分子と分母から分数を返す\lstinline{make-rat}, 分数から分子を返す\lstinline{numer},
分数から分母を返す\lstinline{denom}という3つのプロシージャが存在するとする.

分数の四則演算は

\begin{align*}
  &\frac{n_1}{d_1} + \frac{n_2}{d_2}  = \frac{n_1d_2 + n_2d_1}{d_1d_2}
  & \frac{n_1}{d_1}\cdot \frac{n_2}{d_2} = \frac{n_1n_2}{d_1d_2}\\
  &\frac{n_1}{d_1} - \frac{n_2}{d_2} = \frac{n_1d_2 - n_2d_1}{d_1d_2}
  & \frac{n_1/d_2}{n_2/d_2} = \frac{n_1d_2}{d_1n_2}\\
  &\frac{n_1}{d_1} = \frac{n_2}{d_2} \Leftrightarrow n_1d_2 = n_2d_1
\end{align*}
\noindent
のように定義できるので, 以上の3つのプロシージャがあれば, 問題なくそれぞれの
演算を問題なく実装できる. 例えば, 足し算は以下のように実装できる.

\begin{lstlisting}[basicstyle=\footnotesize,title=分数の足し算]
(define (add-rat x y)
  (make-rat (+ (* (numer x) (denom y))
               (* (numer y) (denom x)))
            (* (denom x) (denom y))))
\end{lstlisting}
\vspace{5mm}

Lispにはペアというデータ構造が存在しており, 2つのデータを1つの構造として表現できる.
\lstinline{cons}で2つの要素から1つのペアを作ることができ, \lstinline{car}で
1つ目の要素を取り出し, \lstinline{cdr}で2つ目の要素を取り出す.
ペアを用いると, 分数を自然に表現することができる.

\begin{lstlisting}[basicstyle=\footnotesize,title=分数の表現]
(define (make-rat n d) (cons n d))
(define (numer x) (car x))
(define (denom x) (cdr x))
\end{lstlisting}

%
\subsubsection{抽象化の壁}
データを扱う時、複数の抽象化のレイヤに分けることが出来る.
分数の例で考えると, 以下の図のようなレイヤが考えられる.
\begin{figure}[ht]
  \centering
  \includegraphics[width=8cm,height=6cm]{imgs/abstraction-barrier.png}
  \caption{\label{fig:abstraction-barrier}有理数における抽象化の壁}
\end{figure}

ここでは抽象化のレベルが4つで分かれている. 最も上のレイヤは分数を扱うプログラムが使うもので,
実際分数はどう表現されているかも, どう四則演算が定義されているのかが分からなくても四則演算が
できるレベルである. その下のレイヤでは, 実際の四則演算の実装である. そこでは分数のセレクタと
コンストラクタを用いるが, その実装に依存しない. その下のレイヤはセレクタとコンストラクタの実装で,
ペアを用いるが,今回もその実装に依存しない. 最も下のレイヤはペアの実装である.

そのレイヤ分けの主な利点はプログラムの保守性と柔軟性の向上である.
下のレイヤの実装に依存しないので, データをどう表現するかが変わらない限り,
実装が変わっても影響が受けない.
図\ref{fig:abstraction-barrier}で有理数における抽象化の壁の例を示す.

%
\subsubsection{データとは}
以上でセレクタとコンストラクタで具体的なデータを抽象的なデータに
変えることがわかったが, 抽象的なデータについて定義を固める必要がある.

\noindent
分数の例において, セレクタについて考える時,
${}^{\forall} ($\lstinline{n}$, $\lstinline{d}$)
\in (\mathbb{N}\times \mathbb{N}\backslash \{0\})$
に対して, 次の関係が満たされる必要がある.
\[
  \text{\lstinline{x = (make-rat n d)}} \Leftrightarrow
  \frac{\text{\lstinline{numer x}}}{\text{\lstinline{denom x}}}
  = \frac{\text{\lstinline{n}}}{\text{\lstinline{d}}}
\]

一般的にデータは正しく表現されるための条件満たしているセレクタとコンストラクタの集合であると言える.


%
\end{document}
