\subsection{ミュータブルなデータ}
章2では, データの抽象化を導入し, コンストラクタとセレクタを用いて,
データを表す方法について説明した. そのデータが変わるような
状態を持つことができることがわかったので, この章ではそのような
ミュータブルなデータについて説明する. そのような抽象データは
コンストラクタとセレクタ以外にも状態を変えるための
ミューテータも持つ. 例えば, 銀行口座の残高を変える
ミューテータであれば以下のような仕様が考えられる.

\begin{lstlisting}[basicstyle=\footnotesize]
(set-balance! <account> <new-value>)
\end{lstlisting}

ミューテータを持ったデータオブジェクトはミュータブルな
データオブジェクトという.
%
\subsubsection{ミュータブルなリスト構造}
ペアに対する操作 (\lstinline{cons}, \lstinline{car}, \lstinline{cdr})は
リストを作る, リストから要素を取り出すために使うことができるが, その操作を用いて
リストを変えることができない. \lstinline{append}や\lstinline{list}も
以上の操作で定義されているので, 同じことが言える. リストを変えるために新しい
操作が必要となる.