\setcounter{subsubsection}{2}
\subsubsection{フレームとローカルステート}
環境モデルを用いると, プロシージャ実行する時にそのプロシージャが
フレームを持つので, そのフレームを用いてローカルの状態を表すことができる.
例えば, 以下の例について考える.

\begin{lstlisting}[basicstyle=\footnotesize,caption=]
(define (make-withdraw balance)
  (lambda (amount)
    (if (>= balance amount)
        (begin (set! balance (- balance amount))
               balance)
        "Insufficient funds")))
(define W1 (make-withdraw 100))
(define W2 (make-withdraw 200))
(W1 20) => 80
(W2 30) => 170
\end{lstlisting}

ここで, \lstinline{make-withdraw}を評価する時に,
新しいフレームが作られて, そのフレームでは\lstinline{balance}が
実引数にバインドされる. その環境の中で, \lstinline{lambda}式が
実行されるので, 作られるプロシージャの環境において, \lstinline{balance}
の値が実引数の値になる. また, \lstinline{make-withdraw}を呼び出すたびに,
同じプロセスでフレームが作られるので, \lstinline{balance}の値は
呼び出しごとにローカルステートとなる.
