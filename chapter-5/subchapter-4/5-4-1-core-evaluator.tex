\subsubsection{明示制御評価機のコア}
評価機の中心は\lstinline{eval-dispatch}で始まる%
命令のシーケンスである. 4.1の\lstinline{eval}プロシジャー%
に相当する部分になっている. そこで, \lstinline{exp}に入っている%
式を\lstinline{env}の環境で実行し, 結果を\lstinline{val}に%
保存する. 実装は次のようにできる.

\begin{lstlisting}[basicstyle=\footnotesize,caption=]
eval-dispatch
  (test (op self-evaluating?) (reg exp))
  (branch (label ev-self-eval))
  (test (op variable?) (reg exp))
  (branch (label ev-variable))
  (test (op quoted?) (reg exp))
  (branch (label ev-quoted))
  (test (op assignment?) (reg exp))
  (branch (label ev-assignment))
  (test (op definition?) (reg exp))
  (branch (label ev-definition))
  (test (op if?) (reg exp))
  (branch (label ev-if))
  (test (op lambda?) (reg exp))
  (branch (label ev-lambda))
  (test (op begin?) (reg exp))
  (branch (label ev-begin))
  (test (op application?) (reg exp))
  (branch (label ev-application))
  (goto (label unknown-expression-type))
\end{lstlisting}

\paragraph{単純式の評価}
サブ式のない式の場合, 結果を\lstinline{val}に入れて, あとは%
\lstinline{continue}に入っているエントリーポイントで%
実行を続ける. 例えば, \lstinline{variable}の場合は次のように%
実装できる.

\begin{lstlisting}[basicstyle=\footnotesize]
ev-variable
  (assign val (op lookup-variable-value) (reg exp) (reg env))
  (goto (reg continue))
\end{lstlisting}
%
\paragraph{プロシジャー適用の評価}
\lstinline{eval}プロシジャーでは, オペレータを評価し,
再帰的にオペランドを評価していき,
結果を\lstinline{apply}に渡すことによって評価していた.
今回も同じアプローチを使い, 再帰は\lstinline{goto}命令で
表現し, 値が必要なレジスタの値を一時的に保存するには, スタックを用いる.
オペレータはの評価は次のように実装できる.

\begin{lstlisting}[basicstyle=\footnotesize]
ev-application
  (save continue)
  (save env)
  (assign unev (op operands) (reg exp))
  (save unev)
  (assign exp (op operator) (reg exp))
  (assign continue (label ev-appl-did-operator))
  (goto (label eval-dispatch))
ev-appl-did-operator
  (restore unev) ; the operands
  (restore env)
  (assign argl (op empty-arglist))
  (assign proc (reg val)) ; the operator
  (test (op no-operands?) (reg unev))
  (branch (label apply-dispatch))
  (save proc)
\end{lstlisting}
